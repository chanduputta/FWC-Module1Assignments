\documentclass[journal,12pt,twocolumn]{article}
\usepackage{graphicx}
\graphicspath{{./figs/}}{}
\usepackage{amsmath,amssymb,amsfonts,amsthm}
\usepackage{gensymb}
\newcommand{\myvec}[1]{\ensuremath{\begin{pmatrix}#1\end{pmatrix}}}

\let\vec\mathbf

\title{
Matrix-Circle
}
\author{SHREYASH CHANDRA PUTTA}
\begin{document}
\maketitle
\tableofcontents
\bigskip
\section{Problem Statement}
To find angle QPR of the triangle PQR which is inscribed in the circle $x^2 + y^2 = 25$. If Q and R  have  co-ordinates (3,4) and (-4,3) respectively .
%\section{}

\begin{table}[h]
    \centering
    \begin{tabular}{|c|c|c|}
       \hline
       \textbf{Symbol}&\textbf{Value}&\textbf{Description}  \\
       \hline
	    $\vec{Q}$ & $\myvec{
		    3\\
		    4}$
	    & given point\\
        \hline
	    $\vec{R}$ & $\myvec{-4\\3}$
 & given point\\
        \hline
	    $\vec{P}$ & $\myvec{0\\-5}$
 & A point on circle  \\
       \hline
 %       a + b & 9 & Given Condition\\
%        \hline
    \end{tabular}
    \caption{Parameters}
    \label{tab:my_label}
\end{table}

\begin{figure}[h]
    \centering
\includegraphics[width=\columnwidth]{fig/circirclefig.pdf}
    \caption{triangle inscribed in Circle and its angle QPR }
    \label{fig:my_label}
\end{figure}
\vspace{2cm}
\section{Solution}
Given that Points given are on the circle forms an inscribed triangle forms by joining the points (3,4),(-4,3) and any point on the given circle   \\
%so, b = 9 - a  \\
\\
Let ${\vec{Q}}$=$\myvec{
  3\\
  4}$
  ,${\vec{R}}$=$\myvec{
  -4\\
  3}$
 and ${\vec{P}}$=$\myvec{
  0\\
  -5}$
\\
\\

%%%%%%%%%%%%%%%%%%%%%%%%%%%%%%%
The direction vector of the line joining two points A, B is given by
\\

\begin{equation}
	\vec{m}=
     \vec{B}-  \vec{A}
  \label{eq-1}
\end{equation}
\\
From eq1 we get dirction vectors of ${\vec{Q},\vec{O}}$ and ${\vec{R},\vec{O}}$ as m1 and m2 respectively 
\\
\begin{equation}
	\vec{m1}=
     \myvec{
  3\\
  4
 }-  \myvec{
  0\\
  -5
 }= \myvec{3\\9}
  \label{eq-2}
\end{equation}
\\
\begin{equation}
	\vec{m2}=
     \myvec{
  -4\\
  3
 }- \myvec{0\\-5} = \myvec{-4\\8}
   \label{eq-3}
\end{equation}
\\
\\
We know,
The normal vector to direction vector is defined by:
\\
\begin{equation}
\vec{m^{\top}}\vec{n} = 0.
  \label{eq-4}
\end{equation} 
\\
From eq4 and eq2 weget
 \\
 \begin{equation}
	\myvec{3\ \ 9}
     \vec{n1}= 0
  \label{eq-5}
\end{equation}
\\
by eq5 we can cansider :
\begin{equation}	
     \vec{n1}= \myvec{-9\\ 3}
  \label{eq-6}
\end{equation}  
\\
From eq4 and eq3 weget
\begin{equation}
	\myvec{-4\ \ 8}
     \vec{n2} = 0
   \label{eq-7}
\end{equation}
\\
by solving eq7 we consider :
\begin{equation}	
     \vec{n2}= \myvec{8\\ 4}
  \label{eq-8}
\end{equation}  
\\

The angle between two vectors is given by
\begin{equation}
\theta = {cos}^{-1}\frac{\vec{n1^{\top}} \vec{n2}}{\vec{\|n1\| \|n2\|}}
   \label{eq-9}
\end{equation}
By substituting $\vec{n1} \:and\: \vec{n2} $ we get
\begin{equation}
 \measuredangle{QPR} = 45\degree
  \label{eq-10}
\end{equation}
\\

where,$ \vec{P} $ is a random point on the given circle
\\


%%%%%%%%%%%%%%%%%%%%%%%%%%%%%%%%%%%%%%%%%%%%%%%%%
 
\section{Software}
Download the following code using,
\begin{table}[h]
    \centering
    \begin{tabular}{|c|}
    \hline \\
         svn co https://github.com/chanduputta/\\FWC-Module1Assignments/blob/\\main/circle/circle.py  \\
         \\
\hline
    \end{tabular}
\end{table}
\\
and execute the code by using command
\begin{center}
\textbf{Python3  circle.py}
\\
\end{center}

\section{Conclusion}
\begin{center}
We found the  $ \measuredangle{QPR} $ of the $\triangle{PQR}$ which is inscribed in the circle $x^2 + y^2 = 25$. Where P is any point on the circle as $45\degree$.
\end{center}
\end{document}
