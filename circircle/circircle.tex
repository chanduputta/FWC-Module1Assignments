\documentclass[journal,12pt,twocolumn]{article}
\usepackage{graphicx}
\graphicspath{{./figs/}}{}
\usepackage{amsmath,amssymb,amsfonts,amsthm}
\usepackage{gensymb}
\newcommand{\myvec}[1]{\ensuremath{\begin{pmatrix}#1\end{pmatrix}}}

\let\vec\mathbf

\title{
Matrix-Circle
}
\author{SHREYASH CHANDRA PUTTA}
\begin{document}
\maketitle
\tableofcontents
\bigskip
\section{Problem Statement}
To find angle QPR of the triangle PQR which is inscribed in the circle $x^2 + y^2 = 25$. If Q and R  have  co-ordinates (3,4) and (-4,3) respectively .
%\section{}

\begin{table}[h]
    \centering
    \begin{tabular}{|c|c|c|}
       \hline
       \textbf{Symbol}&\textbf{Value}&\textbf{Description}  \\
       \hline
	    $\vec{Q}$ & $\myvec{
		    3\\
		    4}$
	    & given point\\
        \hline
	    $\vec{R}$ & $\myvec{-4\\3}$
 & given point\\
        \hline
	    $\vec{P}$ & $\myvec{0\\-5}$
 & A point on circle  \\
       \hline
 %       a + b & 9 & Given Condition\\
%        \hline
    \end{tabular}
    \caption{Parameters}
    \label{tab:my_label}
\end{table}

\begin{figure}[h]
    \centering
\includegraphics[width=\columnwidth]{fig/circirclefig.pdf}
    \caption{triangle inscribed in Circle and its angle QPR }
    \label{fig:my_label}
\end{figure}
\vspace{2cm}
\section{Solution}
Given that Points given are on the circle forms an inscribed triangle forms by joining the points (3,4),(-4,3) and any point on the given circle   \\
%so, b = 9 - a  \\

The circle given,
 \begin{align}
    \label{eq:conic_quad_form}
    \vec{x}^{\top}\vec{x}=25
 \end{align}
\\
\\
General Equation:
\\
\begin{align}
    \label{eq:conic_quad_form}
    \vec{x}^{\top}\vec{V}\vec{x}+2\vec{u}^{\top}\vec{x}+f=0
    \end{align}
where
\begin{align}
	\label{eq:V_matrix}
	\vec{V} &= \myvec{1 & 0\\0 & 1} or \vec{I},
	\\
	\label{eq:u_vector}
	\vec{u} &= -\myvec{0\\0},
	\\
	\label{eq:f_value}
	f &= -25	
\end{align}
%%%%%%%%%%%%%%%%%%%%%%%%%%%%%%%
Radius and centre of circle is :
\begin{align}
	\label{eq:V_matrix}
	\vec{O} = -\vec{u} = \myvec{0\\0}\\
	r = \sqrt{\vec{u}^{\top}\vec{u}-f} = 5
	 \label{eq-1} 
\end{align}
%%%%%%%%%%%%%%%%%%%%%%%%%%%%
\\
Point on the circle should satisfy the circle equation so, we take 
\begin{align}
\vec{P}=\myvec{0\\
  -r}=\myvec{0\\-5}
 \\\vec{Q}=\myvec{3\\
  4}
  \\ \vec{R}=\myvec{-4\\
  3}
 \end{align}
\\
Considering $\vec{P}$,$\vec{Q}$and $\vec{R}$ as the Coordinates of the Inscribed Triangle
\\
\\
We know,
The direction vector of the line joining two points $\vec{A}$, $\vec{B}$ is given by
\begin{equation}
	\vec{m}=
     \vec{B}-  \vec{A}
  \label{eq-11}
\end{equation}
From eq11 we get dirction vectors of $\overline{\vec{P}\vec{Q}}$ as $\vec{m1}$ i.e
\\
\begin{equation}
	\vec{m1}=
     \myvec{
  3\\
  4
 }-  \myvec{
  0\\
  -5
 }= \myvec{3\\9}
  \label{eq-12}
\end{equation}
\\
From eq11 we get dirction vector of $\overline{\vec{P}\vec{R}}$ as $\vec{m2}$ i.e
\begin{equation}
	\vec{m2}=
     \myvec{
  -4\\
  3
 }- \myvec{0\\-5} 
 = \myvec{-4\\8}
   \label{eq-13}
\end{equation}

We know,
The normal vector to direction vector is defined by:
\begin{equation}
\vec{m^{\top}}\vec{n} = 0.
  \label{eq-14}
\end{equation} 

i.e
\\
If $\vec{m} = \myvec{a\\b}$ then $\vec{n} = \myvec{b\\-a}$
\\
\\
From eq12 and eq14 weget
 \\
 \begin{equation}
	\myvec{3\ \ 9}
     \vec{n1}= 0
  \label{eq-15}
\end{equation}
\\
by eq15 we can cansider :
\begin{equation}	
     \vec{n1}= \myvec{9\\ -3}
  \label{eq-16}
\end{equation}  
\\
From eq13 and eq14 weget
\begin{equation}
	\myvec{-4\ \ 8}
     \vec{n2} = 0
   \label{eq-17}
\end{equation}
by solving eq17 we consider :
\begin{equation}	
     \vec{n2}= \myvec{8\\ 4}
  \label{eq-18}
\end{equation}  

The angle between two vectors is given by
\begin{equation}
\theta = {cos}^{-1}\frac{\vec{n1^{\top}} \vec{n2}}{\vec{\|n1\| \|n2\|}}
   \label{eq-19}
\end{equation}
By substituting $\vec{n1} \:and\: \vec{n2} $ we get
\begin{equation}
 \measuredangle{QPR} = 45\degree
  \label{eq-20}
\end{equation}

where,$ \vec{P} $ is a random point on the given circle
%%%%%%%%%%%%%%%%%%%%%%%%%%%%%%%%%%%%%%%%%%%%%%%%%
 
\section{Software}
Download the following code using,
\begin{table}[h]
    \centering
    \begin{tabular}{|c|}
    \hline \\
         svn co https://github.com/chanduputta/\\FWC-Module1Assignments/blob/\\main/circle/circle.py  \\
         \\
\hline
    \end{tabular}
\end{table}
\\
and execute the code by using command
\begin{center}
\textbf{Python3  circle.py}
\\
\end{center}

\section{Conclusion}
\begin{center}
We found the  $ \measuredangle{QPR} $ of the $\triangle{PQR}$ which is inscribed in the circle $x^2 + y^2 = 25$. Where P is any point on the circle as $45\degree$.
\end{center}
\end{document}
