\documentclass[journal,12pt,twocolumn]{article}
\usepackage{graphicx}
\graphicspath{{./figs/}}{}
\usepackage{amsmath,amssymb,amsfonts,amsthm}
\usepackage{gensymb}
\newcommand{\myvec}[1]{\ensuremath{\begin{pmatrix}#1\end{pmatrix}}}
\let\vec\mathbf
\title{
Matrix-Conic
}
\author{SHREYASH CHANDRA PUTTA}
\begin{document}
\maketitle
\tableofcontents
\bigskip
\section{Problem Statement}
To find the locus of mid point of $\overline{PQ}$ where $\vec{P}$ is $\myvec{ 1\\0}$ and $\vec{Q}$ is a point on the locus $y^{2} = 8x$ .
%\section{}

\begin{table}[h]
    \centering
    \begin{tabular}{|c|c|c|}
       \hline
       \textbf{Symbol}&\textbf{Value}&\textbf{Description}  \\
       \hline
	    $\vec{P}$ & $\myvec{
		    1\\
		    0}$
	    & given point\\
        \hline
	    $\vec{Q}$ & $\myvec{x'\\y'}$
 &  point on given locus \\
 
        \hline
	    $\vec{X}$ & $\myvec{x\\y}$
 & mid point of $\overline{PQ}$  \\
       \hline
 %       a + b & 9 & Given Condition\\
%        \hline
    \end{tabular}
    \caption{Parameters}
    \label{tab:my_label}
\end{table}


\section{Solution}
Let $\vec{X}$ be any point on the Locus formed by the midpoint joining the point $\vec{P}$ and any point on the given locus say, point $\vec{Q}$  \\

Where, 
  ${\vec{P}}$=$\myvec{
  1\\
  0}$
  ,${\vec{Q}}$=$\myvec{
  x'\\
  y'}$
 and ${\vec{X}}$=$\myvec{
  x\\
  y}$

\begin{figure}[h]
    \centering
\includegraphics[width=\columnwidth]{fig/conicfig.pdf}
    \caption{Found the locus equation }
    \label{fig:my_label}
\end{figure}
\vspace{0cm}
The given equation of parabola $y^2 = 8x$ can be written in the general quadratic form as
\begin{align}
    \label{eq:conic_quad_form}
    \vec{x}^{\top}\vec{V}\vec{x}+2\vec{u}^{\top}\vec{x}+f=0
    \end{align}
where
\begin{align}
	\label{eq:V_matrix}
	\vec{V} &= \myvec{0 & 0\\0 & 1},
	\\
	\label{eq:u_vector}
	\vec{u} &= \myvec{-4\\0},
	\\
	\label{eq:f_value}
	f &= 0
	%\\
\end{align}
\\
Substitute $\vec{Q}$ and data in eq1 .
\\
\begin{align}
    \label{eq:conic_quad_form}
    \vec{Q}^{\top}\vec{V}\vec{Q}+2\vec{u}^{\top}\vec{Q}=0
    \end{align}

By section formula
mid point of line joining $\vec{P}$ and $\vec{Q}$ as $\vec{X}$ is:
 
 \begin{equation}
	\vec{X}=\frac{\vec{Q}+\vec{P}}{2}
	 \label{eq-4}
\end{equation}
\\
 \begin{equation}
	\vec{Q}=2\vec{X}-\vec{P}
	 \label{eq-4}
\end{equation}
\\
%%%%%%%%%%%%%%%%%%%%%%%%%%%%%%%
From eq5 and eq7 We get 
\\

\begin{multline}
    (2\vec{X}-\vec{P})^{\top}\vec{V}(2\vec{X}-\vec{P})\\+2\vec{u}^{\top}(2\vec{X}-\vec{P})=0
     \label{eq-5}
    \end{multline}

\begin{multline}
    \label{eq:conic_quad_form}
    (2\vec{X}^{\top}\vec{V}-\vec{P}^{\top}\vec{V})(2\vec{X}-\vec{P})\\+2\vec{u}^{\top}2\vec{X}-2\vec{u}^{\top}\vec{P}=0
    \end{multline}

\begin{multline}
    \label{eq:conic_quad_form}
    (2\vec{X}^{\top}\vec{V}2\vec{X}-2\vec{X}^{\top}\vec{V}\vec{P})\\+2\vec{u}^{\top}2\vec{X}-2\vec{u}^{\top}\vec{P}=0
    \end{multline}
    
\begin{multline}
    \label{eq:conic_quad_form}
    4\vec{X}^{\top}\vec{V}\vec{X}+4\vec{u}^{\top}\vec{X}+8=0
    \end{multline}

%%%%%%%%%%%%%%%%%%%%%%%%%%%%%%%%%%%%%%%%%%%%%%%%%%%%%%%%

Therefore, required Locus equation of the mid point of given point $\vec{P}$ and $\vec{Q}$ is obtained as:
\begin{multline}
    \label{eq:conic_quad_form}
    \vec{X}^{\top}\vec{V}\vec{X}+\vec{u}^{\top}\vec{X}+2=0
    \end{multline}
%%%%%%%%%%%%%%%%%%%%%%%%%%%%%%%

%%%%%%%%%%%%%%%%%%%%%%%%%%%%%%%%%%%%%%%%%%%%%%%%%
 
\section{Software}
Download the following code using,
\begin{table}[h]
    \centering
    \begin{tabular}{|c|}
    \hline \\
         svn co https://github.com/chanduputta/\\FWC-Module1Assignments/blob/\\main/conic/code/conic.py \\
         \\
\hline
    \end{tabular}
\end{table}
\\
and execute the code by using command
\begin{center}
\textbf{cmd:}Python3  conic.py
\\
\end{center}

\section{Conclusion}
\begin{center}
We found the locus of mid point of $\overline{PQ}$ where $\vec{P}$ is $\myvec{ 1\\0}$ and $\vec{Q}$ is a point on the locus $y^{2} = 8x$ as $y^{2} = 4x - 2$.
\end{center}
\end{document}
