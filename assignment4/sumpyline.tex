\documentclass[journal,12pt,twocolumn]{article}
\usepackage{graphicx}
\graphicspath{{./figs/}}{}
\usepackage{amsmath,amssymb,amsfonts,amsthm}
\newcommand{\myvec}[1]{\ensuremath{\begin{pmatrix}#1\end{pmatrix}}}

\let\vec\mathbf

\title{
Matrix-Lines
}
\author{SHREYAS CHANDUPUTTA}
\begin{document}
\maketitle
\tableofcontents
\bigskip
\section{Problem Statement}
A line perpendicular to the line segement joining the points (1,0)and(2,3)divides it in the ratio 1:n . find the equation of the line? .\\
\section{Construction}
\begin{figure}[h]
    \centering
%\includegraphics[width=\columnwidth]{figs/assign4.png}
    \caption{Equation of the Straight Line}
    \label{fig:my_label}
\end{figure}
\vspace{2cm}
\begin{table}[h]
    \centering
    \begin{tabular}{|c|c|c|}
       \hline
       \textbf{Symbol}&\textbf{Value}&\textbf{Description}  \\
       \hline
	    $\vec{P}$ & $\myvec{
		    1\\
		    0}$
	    & given point\\
        \hline
	    $\vec{Q}$ & $\myvec{2\\3}$
 & given point\\
        \hline
	    $\vec{R}$ & $\myvec{
  \frac{2+n}{1+n}\\
  \frac{3}{1+n}}$
 & intersecting point  \\
       \hline
 %       a + b & 9 & Given Condition\\
%        \hline
    \end{tabular}
    \caption{Parameters}
    \label{tab:my_label}
\end{table}


\section{Solution}
Given that resultant will divide the equation of line in the ratio 1:n and the line is perpendicular to line segment joining the points (1,0)and(2,3)  ) \\
%so, b = 9 - a  \\
\\
Let ${\vec{P}}$=$\myvec{
  1\\
  0}$
 and ${\vec{Q}}$=$\myvec{
  2\\
  3}$
\\
\\
Equation of line is ${\vec{n^{\top}}\vec{X}} = c$.
\\
\\
We know if 2 points of the linesegment is given then,\\
 %The Equation of line through ${\vec{P}}$ is\\
%\begin{equation}
%	\vec{n^{\top}}
%	\myvec{
 % a\\
  %0}
  %= c \label{eq-1}
%\end{equation}
\\
Direction vector of line joining two points  ${\vec{P}}$ ${\vec{Q}}$ is M=Q-P\\

\begin{equation}
	\vec{M}=
     \myvec{
  2\\
  3
 }-  \myvec{
  1\\
  0
 }
  \label{eq-2}
\end{equation}
\\
\\
\begin{equation}
	\vec{M}=
     \myvec{
  1\\
  3
 }
   \label{eq-2}
\end{equation}
\\
We know, that position or  directional vector of points P and Q line segement used as the normal vector
\\
\\
 The general equation of the required perpendicular line is
 ${\vec{M^{\top}}\vec{X}} = c$.
 \\
 \\
 The perpendicular line cutting a line segment P and Q in ratio 1:n is passes through the point R.
 
 \begin{equation}
	\vec{R}=\frac{Q+nP}{1+n}
	 \label{eq-4}
\end{equation}
 \\
Equation of line passing through ${\vec{R}}$ is\\
\begin{equation}
	\vec{M^{\top}}(X-R)=0
	 \label{eq-4}
\end{equation}
\\
\begin{equation}
	 \vec{M^{\top}}
	 \vec{X} - \vec{M^{\top}}
	 \vec{R} = 0
	 \label{eq-5}
\end{equation}
 \\
 From eq3, eq5 and eq2 we can find the required Perpenducular line equation. 
 \begin{equation}
	   \myvec{
  1\\
  3}^{\top}
	 \vec{X}
	 = \myvec{
  1\\
  3}^{\top}\myvec{
  \frac{2+n}{1+n}\\
  \frac{3}{1+n}} 
  \label{eq-5}
\end{equation}

 %\textbf{
Therefore, equation of the line is :
\begin{center}
    \textbf{nx+1x+3y+3ny = n+11}
\end{center}


 
\section{Software}
Download the following code using,
\begin{table}[h]
    \centering
    \begin{tabular}{|c|}
    \hline \\
         svn co https://github.com/\\mygit-sampath-govardhan/fwc-iith-assignments/blob/\\5b65abbf8e5e3c803b1bff8cf4a95092e100de75/\\Assignment-4(Matrices-line)/codes/Assignment4.py  \\
         \\
\hline
    \end{tabular}
\end{table}
\\
and execute the code by using command
\begin{center}
\textbf{Python3  Assignment4.py}\\
\end{center}

\section{Conclusion}
\begin{center}
We found the equation of a line perpendicular to the line segement joining the points (1,0)and(2,3) divides it in the ratio 1:n .
\end{center}
\end{document}
