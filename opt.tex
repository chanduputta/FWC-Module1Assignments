\documentclass[journal,10pt,twocolumn]{article}
\usepackage[margin=0.5in]{geometry}
\usepackage[cmex10]{amsmath}
\usepackage{amsmath}
\usepackage{array}
\usepackage{booktabs}
% The preceding line is only needed to identify funding in the first footnote. If that is unneeded, please comment it out.
\usepackage{cite}
\usepackage{amsmath,amssymb,amsfonts}
\usepackage{graphicx}
\usepackage{textcomp}
\usepackage{xcolor}
\graphicspath{{./figs/}}{}
\usepackage{amsmath,amssymb,amsfonts,amsthm}
\usepackage{gensymb}
\newcommand{\myvec}[1]{\ensuremath{\begin{pmatrix}#1\end{pmatrix}}}
\let\vec\mathbf
\title{
Optimization-Assignment
}
\author{SHREYASH CHANDRA PUTTA}
\providecommand{\norm}[1]{\left\lVert#1\right\rVert}
\providecommand{\abs}[1]{\left\vert#1\right\vert}
\let\vec\mathbf


\begin{document}
\maketitle
\tableofcontents
\bigskip
\section{Problem Statement}
If x and y are positive real numbers such that $x^2 +y^2 = 1$ then Find the maximum value of \textbf(x+y)

\section{Solution}

\begin{enumerate}

 \item Deriving f(x): 
 Given, equation 
	\begin{align}
      x^2+y^2 = 1 \label{eq:1}
    \end{align}
     we take $(\ref{eq:1})$ as
    \begin{align}
      y = \sqrt{1-x^2} \label{eq:2}
    \end{align}
    where, $x > 0 \: and \: y > 0$
	\\ \\And, given conditon to find maximum of
	\begin{align}
          x+y \label{eq:3}
    \end{align}
From \ref{eq:2} and \ref{eq:3} we get, 
  \begin{align}
      f(x) = x+\sqrt{1-x^2} \label{eq:4}
    \end{align}
    \\ 
\item For Maxima : \\
    Using gradient ascent method,
\begin{align}
    x_n=x_{n-1}+\mu\frac{df(x)}{dx} \label{eq:9}
    \end{align}
    \begin{align}
    \frac{df(x)}{dx}=1-\frac{x}{\sqrt{1-x^2}} \label{eq:10}
\end{align}
After substituting \ref{eq:10} in \ref{eq:9} we get:
\begin{align}
    x_n=x_{n-1}+\mu(1-\frac{x}{\sqrt{1-x^2}})\label{eq:11}
\end{align}
Taking $x_0 = 0.5, \mu = 0.001$ and $precision = 0.00000001$, values obtained using python are:
\begin{align}
\boxed{\text{Maxima} = 1.4142 \approx \sqrt{2}} \\
\boxed{\text{Maxima Point} = 0.7071 \approx \frac{1}{\sqrt{2}}}
\end{align}
\end{enumerate}

 \section{Plot}
\includegraphics[width=1\columnwidth]{fig/optfig.pdf}

\centering \text{Graph of f(x) = $x^2+\sqrt{1-x^2}$}
%%%%%%%%%%%%%%%%%%%%%%%%%%%%%%%%%%%%%%%%%%%%%%%%%%%%%%%%%%%

\end{document}
